\documentclass{sig-alternate-ipsn13}

\begin{document}

\title{Reactive Scheduling of Computational \\ Resources in Control Systems}
%
% You need the command \numberofauthors to handle the 'placement
% and alignment' of the authors beneath the title.
%
% For aesthetic reasons, we recommend 'three authors at a time'
% i.e. three 'name/affiliation blocks' be placed beneath the title.
%
% NOTE: You are NOT restricted in how many 'rows' of
% "name/affiliations" may appear. We just ask that you restrict
% the number of 'columns' to three.
%
% Because of the available 'opening page real-estate'
% we ask you to refrain from putting more than six authors
% (two rows with three columns) beneath the article title.
% More than six makes the first-page appear very cluttered indeed.
%
% Use the \alignauthor commands to handle the names
% and affiliations for an 'aesthetic maximum' of six authors.
% Add names, affiliations, addresses for
% the seventh etc. author(s) as the argument for the
% \additionalauthors command.
% These 'additional authors' will be output/set for you
% without further effort on your part as the last section in
% the body of your article BEFORE References or any Appendices.

\numberofauthors{2} %  in this sample file, there are a *total*
% of EIGHT authors. SIX appear on the 'first-page' (for formatting
% reasons) and the remaining two appear in the \additionalauthors section.
%
\author{
% You can go ahead and credit any number of authors here,
% e.g. one 'row of three' or two rows (consisting of one row of three
% and a second row of one, two or three).
%
% The command \alignauthor (no curly braces needed) should
% precede each author name, affiliation/snail-mail address and
% e-mail address. Additionally, tag each line of
% affiliation/address with \affaddr, and tag the
% e-mail address with \email.
%
% 1st. author
    \alignauthor Gera Weiss\\
       \affaddr{Ben-Gurion University}\\
       \affaddr{Beer-Sheva, Israel}\\
       \email{geraw@cs.bgu.ac.il}
% 2th. author
    \alignauthor Hodai Goldman\\
        \affaddr{Ben-Gurion University}\\
        \affaddr{Beer-Sheva, Israel}\\
        \email{hodaig@cs.bgu.ac.il}
}

\maketitle
\begin{abstract}

\end{abstract}

\section{Introduction}

Cyber-physical systems, where computers interact with and control physical entities ar at the heart of many critical applications. In a typical cyber-physical application, such as robot control, there are many control loops, responsible for various aspects of the system, that run simultaneously and share the same computational resources. Thanks to the ever growing computational power, engineer often tend to ignore this shared resource and design the controllers as if there is 


\section{Architecture: Automata Based Scheduler}

Sections go here.

\section{Simulations}
\label{sec:simulation}
%TODO - show some matlab simulations of hybrid systems.

\section{Application to Autonomous Quad-rotor Flying In-Door}
% overview of why we use vision example

TODO - The quadrotor basics

%TODO - Window test case problem
The specific case study we used to test our concept is the ability of fly thru windows, this task require a very stable hovering in front of the window.
We implement Autonomous controller for that task, and we evaluate the performance in terms of the horizontal linear axis parallel to the window surface, as this is the most critical axis in this task, this axis noted by $x$ see Figure~\ref{fig:window axis ???}.

%controller and vision algorithms (appendix?)
The drone consist of Raspberry pi with navio~\cite{??? raspberry, navio} and is controlled by modified ArduPilot~\cite{??? APM}.
We implement standard PID~\cite{??? PID} controller that regulate the position and attitude of the drone relatively to the window.
We used inertial sensors (Gyroscope and Accelerometer) for attitude control relative to earth, and we use camera~\cite{??? picam} and simple image processing for position and attitude relatively to the window~\ref{sec:visionAlgo ???}.
The image processing have few different operation modes which have different image resolution, better resolution image produce more accurate measurement but consume more processing time, changing operation modes allows us to control the trade off between mesurment quality and processing time as shown in Table~\ref{tab:tradeoff ???}, for simplicity we examine only two modes, \textit{High quality} with resolution of 960p and \textit{Low quality} with resolution of 240p.

%TODO - state estimator, Kalman approximation (what should be explained here?)
%TODO - http://robottini.altervista.org/kalman-filter-vs-complementary-filter
%TODO - maybe we can cite this: http://proyectos.ciii.frc.utn.edu.ar/cuadricoptero/export/262a5075a7aa6575102f84fcdc2ee39ecab3c530/documentacion/articulo_case_filtro_comp/referencias/A%20comparison%20of%20complementary%20and%20kalman%20filtering.pdf
As shown in the simulation at Section~\ref{sec:simulation} we should use Kalman filter estimator, in our case we have non-linear system... %TODO - continue
For simplicity, inspiring from Kalman filter~\cite{??? kalman}, we use two steps estimator that \textit{predict} the current state evolved from the previous state using the system model ($\hat{x_{k|k-1}}$) and then \textit{update} the prediction with current state measurement from image ($z_k$).
The result estimation ($\hat{x_{k|k}}$) is weighted average: 
$ \hat{x_{k|k}} = K \hat{x_{k|k-1}} + (1-K) z_k $, 
the $K$ gain is defined different for each resolution for achieving estimation in each resolution.
%TODO - need to explane exactly which variables we mesure or what is the state vector??

\subsection{specification automatons}
%TODO - the schedulers we tested

The performances of the system is measured by 

TODO - only High vs only Low (with performances results)

TODO - error ($y_{k|k}$) derived schedulers

TODO - position ($x_{k|k}$) derived schedulers

TODO - complex / agregated schedulers

TODO - gyro derived??


\subsubsection{results}
TODO - graphs and tables


\section{Conclusions}
Conclusion goes here.


%ACKNOWLEDGMENTS are optional
\section*{Acknowledgments}
Acknowledgement goes here.


%
% The following two commands are all you need in the
% initial runs of your .tex file to
% produce the bibliography for the citations in your paper.
\bibliographystyle{abbrv}
\bibliography{reactive_scheduler}  % reactive_scheduler.bib is the name of the Bibliography in this case
% You must have a proper ".bib" file
%  and remember to run:
% latex bibtex latex latex
% to resolve all references
%
% ACM needs 'a single self-contained file'!
%
%APPENDICES are optional
%\balancecolumns
\appendix
%Appendix A

Appendix goes here.

% That's all folks!
\end{document}
