\documentclass{sig-alternate-ipsn13}

\begin{document}

\title{reactive scheduler: \ttlit{Computational resource management of multi channel controller}}
%
% You need the command \numberofauthors to handle the 'placement
% and alignment' of the authors beneath the title.
%
% For aesthetic reasons, we recommend 'three authors at a time'
% i.e. three 'name/affiliation blocks' be placed beneath the title.
%
% NOTE: You are NOT restricted in how many 'rows' of
% "name/affiliations" may appear. We just ask that you restrict
% the number of 'columns' to three.
%
% Because of the available 'opening page real-estate'
% we ask you to refrain from putting more than six authors
% (two rows with three columns) beneath the article title.
% More than six makes the first-page appear very cluttered indeed.
%
% Use the \alignauthor commands to handle the names
% and affiliations for an 'aesthetic maximum' of six authors.
% Add names, affiliations, addresses for
% the seventh etc. author(s) as the argument for the
% \additionalauthors command.
% These 'additional authors' will be output/set for you
% without further effort on your part as the last section in
% the body of your article BEFORE References or any Appendices.

\numberofauthors{2} %  in this sample file, there are a *total*
% of EIGHT authors. SIX appear on the 'first-page' (for formatting
% reasons) and the remaining two appear in the \additionalauthors section.
%
\author{
% You can go ahead and credit any number of authors here,
% e.g. one 'row of three' or two rows (consisting of one row of three
% and a second row of one, two or three).
%
% The command \alignauthor (no curly braces needed) should
% precede each author name, affiliation/snail-mail address and
% e-mail address. Additionally, tag each line of
% affiliation/address with \affaddr, and tag the
% e-mail address with \email.
%
% 1st. author
    \alignauthor Gera Weiss\\
       \affaddr{Ben-Gurion University}\\
       \affaddr{Beer-Sheva, Israel}\\
       \email{geraw@cs.bgu.ac.il}
% 2th. author
    \alignauthor Hodai Goldman\\
        \affaddr{Ben-Gurion University}\\
        \affaddr{Beer-Sheva, Israel}\\
        \email{hodaig@cs.bgu.ac.il}
}

\maketitle
\begin{abstract}

%TODO - hodai - just copy from the thesis/proposal, need to rewrite
Today's computer power allows for consolidation of controllers towards systems where a single computer regulates many control loops, each with its varying needs of computation resources.
This brings two research challenges that we intend to attack in this thesis:
\begin{itemize}
    \item How to schedule control tasks in order to achieve good performance in terms of control measures (overshoot, convergence time, etc.), within the still limited computation resources?
    \item What is a good interface for co-design of scheduling and control?
\end{itemize}

Desktop type operating systems, like Windows and Linux, schedule for computational efficiency but do not allow for worst-case performance guarantees. Real-time operating systems, on the other hand, sacrifice some efficiency for timing predictability, but the type of timing guarantees that such systems provide usually are not directly guaranteeing the control performance of the system. 
When using such operating systems for control, engineers usually apply controllers that work in a fixed periodic manner witch the control behavior becomes deterministic and control performance can be guaranteed. This is not efficient because resources can be better utilized if controllers act at higher frequencies only when needed.

In this work we show how to combine the efficiency of dynamic scheduling with the predictability of real-time scheduling, in a way that is more suitable for control systems then periods and deadlines.
We show that applying control computations at dynamically adjustable periodic, and with variable computational demands, based on real-time information, allows for better utilization of the computational resources and therefor better control performance.

\end{abstract}

\section{Introduction}

Sections go here.

\section{Architecture: Automata Base Scheduler}

Sections go here.

\section{Proof of Concept}
\label{sec:concept}
%TODO - explain that we developing an idea that culd in the future become a tool

\subsection{Simulations}
%TODO - show some matlab simulations of hybrid systems.

\subsection{Experiment: Autonomous Quad-rotor Flying In-Door}
% overview of why we use vision example



\section{Conclusions}
Conclusion goes here.


%ACKNOWLEDGMENTS are optional
\section*{Acknowledgments}
Acknowledgement goes here.

%
% The following two commands are all you need in the
% initial runs of your .tex file to
% produce the bibliography for the citations in your paper.
\bibliographystyle{abbrv}
\bibliography{reactive_scheduler}  % reactive_scheduler.bib is the name of the Bibliography in this case
% You must have a proper ".bib" file
%  and remember to run:
% latex bibtex latex latex
% to resolve all references
%
% ACM needs 'a single self-contained file'!
%
%APPENDICES are optional
%\balancecolumns
\appendix
%Appendix A

Appendix goes here.

% That's all folks!
\end{document}
