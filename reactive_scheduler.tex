\documentclass{sig-alternate-ipsn13}

\begin{document}

\title{Reactive Scheduling of Computational \\ Resources in Control Systems}
%
% You need the command \numberofauthors to handle the 'placement
% and alignment' of the authors beneath the title.
%
% For aesthetic reasons, we recommend 'three authors at a time'
% i.e. three 'name/affiliation blocks' be placed beneath the title.
%
% NOTE: You are NOT restricted in how many 'rows' of
% "name/affiliations" may appear. We just ask that you restrict
% the number of 'columns' to three.
%
% Because of the available 'opening page real-estate'
% we ask you to refrain from putting more than six authors
% (two rows with three columns) beneath the article title.
% More than six makes the first-page appear very cluttered indeed.
%
% Use the \alignauthor commands to handle the names
% and affiliations for an 'aesthetic maximum' of six authors.
% Add names, affiliations, addresses for
% the seventh etc. author(s) as the argument for the
% \additionalauthors command.
% These 'additional authors' will be output/set for you
% without further effort on your part as the last section in
% the body of your article BEFORE References or any Appendices.

\numberofauthors{2} %  in this sample file, there are a *total*
% of EIGHT authors. SIX appear on the 'first-page' (for formatting
% reasons) and the remaining two appear in the \additionalauthors section.
%
\author{
% You can go ahead and credit any number of authors here,
% e.g. one 'row of three' or two rows (consisting of one row of three
% and a second row of one, two or three).
%
% The command \alignauthor (no curly braces needed) should
% precede each author name, affiliation/snail-mail address and
% e-mail address. Additionally, tag each line of
% affiliation/address with \affaddr, and tag the
% e-mail address with \email.
%
% 1st. author
    \alignauthor Gera Weiss\\
       \affaddr{Ben-Gurion University}\\
       \affaddr{Beer-Sheva, Israel}\\
       \email{geraw@cs.bgu.ac.il}
% 2th. author
    \alignauthor Hodai Goldman\\
        \affaddr{Ben-Gurion University}\\
        \affaddr{Beer-Sheva, Israel}\\
        \email{hodaig@cs.bgu.ac.il}
}

\maketitle
\begin{abstract}

\end{abstract}

\section{Introduction}

%TODO - hodai - just copy from the thesis/proposal, need to rewrite
Today's computer power allows for consolidation of controllers towards systems where a single computer regulates many control loops, each with its varying needs of computation resources.
This brings two research challenges that we tackle in this work: 
\begin{itemize}
	\item How to schedule control tasks in order to achieve good performance in terms of control measures (overshoot, convergence time, etc.), within the still limited computation resources?
	\item What is a good interface for co-design of scheduling and control?
\end{itemize}

Desktop type operating systems, like Windows and Linux, schedule for computational efficiency but do not allow for worst-case performance guarantees. Real-time operating systems, on the other hand, sacrifice some efficiency for timing predictability, but the type of timing guarantees that such systems provide usually are not directly guaranteeing the control performance of the system. 
When using such operating systems for control, engineers usually apply controllers that work in a fixed periodic manner witch the control behavior becomes deterministic and control performance can be guaranteed. This is not efficient because resources can be better utilized if controllers act at higher frequencies only when needed.

In this work we show how to combine the efficiency of dynamic scheduling with the predictability of real-time scheduling, in a way that is more suitable for control systems then periods and deadlines.
We show that applying control computations at dynamically adjustable periodic, and with variable computational demands, based on real-time information, allows for better utilization of the computational resources and therefor better control performance.

\section{Architecture: Automata Based Scheduler}

Sections go here.

\section{Simulations}
\label{sec:simulation}
%TODO - show some matlab simulations of hybrid systems.

\section{Proposed Methodology}
\begin{enumerate}
    
    \item \textbf{Controller design:} Based on the separation principle, we propose to design the controller to achieve the control objectives assuming a perfect observation. In practice, this may not be feasible because controller designs such as PID require a system to experiment with. In this case, as demonstrated in the case-study below (see Section~\ref{sec:caseStady}), we propose to work with one of the observation modes. If the system is close to linear, this should result with a near optimal design.
    
    \item \textbf{Observers design:} Specification of sensor modes and observers design
    
    \item \textbf{Performance analysis:} Now, we can perform some experiments with the different observer modes and analyze transient behaviors. Specifically, as shown in the case study~\ref{sec:caseStady_analysis}, we can measure how long it takes for the error to accumulate after switching to a lesser observation mode and formulate how this error affects the control objectives.

    \item \textbf{Scheduling automata design:} Base on the analysis we can specify the resource scheduling requirements in the form of \textit{specification automaton}. The goal is to design flexible specification that allow dynamic scheduling in order to adapt the environment and the system state, this will improve the system efficient.

\end{enumerate}


\section{Application to Autonomous Quad-rotor Flying In-Door}
\label{sec:caseStady}
% overview of why we use vision example


TODO - The quadrotor basics

%TODO - Window test case problem
The specific case study we used to test our concept is the ability of fly thru windows, this task require a very stable hovering in front of the window.
We implement Autonomous controller for that task, and we evaluate the performance in terms of the horizontal linear axis parallel to the window surface, as this is the most critical axis in this task, this axis noted by $x$ see Figure~\ref{fig:window axis ???}.

\subsection{Controller Design}
%TODO - split controller & observer
%controller and vision algorithms (appendix?)
The drone consist of Raspberry pi with navio~\cite{??? raspberry, navio} and is controlled by modified ArduPilot~\cite{??? APM}.
We implement standard PID~\cite{??? PID} controller that regulate the position and attitude of the drone relatively to the window.
We used inertial sensors (Gyroscope and Accelerometer) for attitude control relative to earth, and we use camera~\cite{??? picam} and simple image processing for position and attitude relatively to the window~\ref{sec:visionAlgo ???}.
The image processing have few different operation modes which have different image resolution, better resolution image produce more accurate measurement but consume more processing time, changing operation modes allows us to control the trade off between mesurment quality and processing time as shown in Table~\ref{tab:tradeoff ???}, for simplicity we examine only two modes, \textit{High quality} with resolution of 960p and \textit{Low quality} with resolution of 240p.

%TODO - state estimator, Kalman approximation (what should be explained here?)
%TODO - http://robottini.altervista.org/kalman-filter-vs-complementary-filter
%TODO - maybe we can cite this: http://proyectos.ciii.frc.utn.edu.ar/cuadricoptero/export/262a5075a7aa6575102f84fcdc2ee39ecab3c530/documentacion/articulo_case_filtro_comp/referencias/A%20comparison%20of%20complementary%20and%20kalman%20filtering.pdf
As shown in the simulation at Section~\ref{sec:simulation} we should use Kalman filter estimator, in our case we have non-linear system... %TODO - continue
For simplicity, inspiring from Kalman filter~\cite{??? kalman}, we use two steps estimator that \textit{predict} the current state evolved from the previous state using the system model ($\hat{x_{k|k-1}}$) and then \textit{update} the prediction with current state measurement from image ($z_k$).
The result estimation ($\hat{x_{k|k}}$) is weighted average: 
$ \hat{x_{k|k}} = K \hat{x_{k|k-1}} + (1-K) z_k $, 
the $K$ gain is defined different for each resolution for achieving estimation in each resolution.
%TODO - need to explane exactly which variables we mesure or what is the state vector??

\subsection{Analysis and Specification Automata}

The objective of the system is to maintain stable hovering in front of the window, and the performances of the system is measured by the amount of deviation from the center line of the window in the critical axis $x$ (see Figure~\ref{fig:window axis ???}), and in other hand we gauge efficiency by the amount of processing resources is used by the heavy observation tasks (CPU percentage).

Analyzing the test results showing in Table~\ref{tab:res ???} we see that High quality observation mode provide 0.4 meter error tolerance meaning we can fly this mode in 1.2 meter wide corridor~\footnote{Experiments was done with 0.4 meter wide quad-rotor}, with the cost of 30\% CPU usage. 
With adaptive scheduling specification we lower the CPU usage without significant worsening of High quality performances.

%TODO - error ($y_{k|k}$) derived schedulers
From the Low quality experiment graph we can see that measurement post-fit residual ($\tilde{y}_{k|k}$) is accumulate proportional to the deviation from the center of $x$ axis, and we can use $\tilde{y}_{k|k}$ vales to predict 


TODO - position ($x_{k|k}$)  derived schedulers

TODO - complex / agregated schedulers

TODO - gyro derived??


\subsubsection{results}
TODO - graphs and tables


\section{Conclusions}
Conclusion goes here.


%ACKNOWLEDGMENTS are optional
\section*{Acknowledgments}
Acknowledgement goes here.


%
% The following two commands are all you need in the
% initial runs of your .tex file to
% produce the bibliography for the citations in your paper.
\bibliographystyle{abbrv}
\bibliography{reactive_scheduler}  % reactive_scheduler.bib is the name of the Bibliography in this case
% You must have a proper ".bib" file
%  and remember to run:
% latex bibtex latex latex
% to resolve all references
%
% ACM needs 'a single self-contained file'!
%
%APPENDICES are optional
%\balancecolumns
\appendix
%Appendix A

Appendix goes here.

% That's all folks!
\end{document}
